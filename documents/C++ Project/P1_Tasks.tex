\documentclass[12pt]{article}
\usepackage[utf8]{inputenc}
\usepackage{amsmath,amssymb}
\usepackage{enumitem}
\usepackage{hyperref}
\usepackage{geometry}
\geometry{margin=1in}
\usepackage{listings}
\lstset{language=C++,basicstyle=\ttfamily,breaklines=true}

\title{Assignment: Peano Arithmetic Synthesizer \\ \small Inductive Math Modeling via Classes and Recursion}
\author{}
\date{}

\begin{document}

\maketitle

\section*{Overview}
This project challenges you to model natural numbers from scratch using Peano's axioms. Designed for learners who value logical rigor (Abstract Sequential) and creative problem-solving (Concrete Random), this assignment will guide you through:
\begin{itemize}[leftmargin=*]
    \item Analyzing underlying principles of inductive reasoning.
    \item Constructing recursive data structures.
    \item Implementing arithmetic operations using clear, logical steps.
    \item Exploring creative performance enhancements.
\end{itemize}

\section{Underlying Principles and Compelling Reasons}
\textbf{Why this project?}
\begin{itemize}[leftmargin=*]
    \item \textbf{Inductive Reasoning:} Understand how the Peano axioms underpin natural number arithmetic.
    \item \textbf{Logical Analysis:} Break down complex ideas into base cases and inductive steps.
    \item \textbf{Creative Exploration:} Experiment with alternative memory management strategies and performance benchmarking.
    \item \textbf{Practical Application:} Bridge theoretical concepts with hands-on C++ coding.
\end{itemize}

\section{Tasks}

\subsection*{Task 1: Define the Peano Axioms in English}
\begin{itemize}[leftmargin=*]
    \item \textbf{Objective:} Write down the Peano axioms for $\mathbb{N}$.
    \item \textbf{Instructions:}
    \begin{itemize}
        \item Identify which statements serve as the base case(s) and which are inductive.
        \item Provide logical reasoning for the structure of each axiom.
    \end{itemize}
    \item \textbf{Goal:} Solidify your understanding of how recursive structures form the basis of natural number arithmetic.
\end{itemize}

\subsection*{Task 2: Define a Recursive Struct for Natural Numbers}
\begin{itemize}[leftmargin=*]
    \item \textbf{Objective:} Create a basic \texttt{PeanoNumber} structure.
    \item \textbf{Requirements:}
    \begin{itemize}
        \item A boolean flag \texttt{isZero} to indicate the base case.
        \item A pointer to another \texttt{PeanoNumber} representing the successor.
    \end{itemize}
    \item \textbf{Implementation:}
    \begin{itemize}
        \item Write a constructor that initializes either a zero or a successor type.
        \item Implement a \texttt{print()} method to display the integer value.
    \end{itemize}
    \item \textbf{Key Concepts:} Recursive data structures, pointer ownership, and proper object construction.
\end{itemize}

\subsection*{Task 3: Write a Recursive Addition Function}
\begin{itemize}[leftmargin=*]
    \item \textbf{Function Signature:} \texttt{PeanoNumber* add(PeanoNumber* a, PeanoNumber* b)}
    \item \textbf{Recursive Definition:}
    \[
    \begin{aligned}
    \texttt{add}(a, 0) &= a, \\
    \texttt{add}(a, S(b)) &= S(\texttt{add}(a, b))
    \end{aligned}
    \]
    \item \textbf{Test Case:} Add 2 and 3; use your \texttt{print()} method to verify the result.
\end{itemize}

\subsection*{Task 4: Implement Multiplication}
\begin{itemize}[leftmargin=*]
    \item \textbf{Function Signature:} \texttt{PeanoNumber* multiply(PeanoNumber* a, PeanoNumber* b)}
    \item \textbf{Recursive Definition:}
    \[
    \begin{aligned}
    \texttt{multiply}(a, 0) &= 0, \\
    \texttt{multiply}(a, S(b)) &= \texttt{add}(a, \texttt{multiply}(a, b))
    \end{aligned}
    \]
    \item \textbf{Test Case:} Multiply 2 and 3 to produce 6.
\end{itemize}

\subsection*{Task 5: Wrap in a Class with Operator Overloads}
\begin{itemize}[leftmargin=*]
    \item \textbf{Class Name:} \texttt{PeanoInt}
    \item \textbf{Requirements:}
    \begin{itemize}
        \item Encapsulate an internal pointer to your recursive structure.
        \item Implement a constructor that accepts an \texttt{int} and builds the corresponding Peano representation.
        \item Provide a \texttt{toInt()} method to convert back to a native integer.
        \item Overload operators: \texttt{+}, \texttt{*}, \texttt{==}, and the stream output operator (\texttt{\textless\textless}).
    \end{itemize}
    \item \textbf{Design Tips:}
    \begin{itemize}
        \item Use smart pointers (e.g., \texttt{std::shared\_ptr}) or implement a custom destructor to ensure proper memory management.
        \item Guard against memory leaks to ensure robust performance.
    \end{itemize}
\end{itemize}

\subsection*{Task 6: Test Your Class}
\begin{itemize}[leftmargin=*]
    \item \textbf{Implementation:} Write a \texttt{main()} function.
    \item \textbf{Test Cases:}
    \begin{itemize}
        \item Validate that \texttt{2 + 3 = 5} and \texttt{2 * 3 = 6}.
        \item Output results using \texttt{std::cout}.
    \end{itemize}
\end{itemize}

\subsection*{Bonus Task: Compile-Time Peano Arithmetic}
\begin{itemize}[leftmargin=*]
    \item \textbf{Objective:} Explore template metaprogramming to encode Peano numbers at compile-time.
    \item \textbf{Instructions:}
    \begin{itemize}
        \item Define \texttt{struct Zero \{\};} and \texttt{template <typename N> struct Succ \{\};}.
        \item Implement \texttt{Add<A, B>::result} using \texttt{typedef}.
    \end{itemize}
    \item \textbf{Goal:} Gain insight into how compile-time logic can represent inductive reasoning, blending theory with practice.
\end{itemize}

\subsection*{Black Diamond: Performance / Systems Challenge}
\begin{itemize}[leftmargin=*]
    \item \textbf{Benchmarking:} Compare the performance of your Peano arithmetic operations with native integer operations.
    \item \textbf{Optimization:}
    \begin{itemize}
        \item Implement a basic allocator for \texttt{PeanoNode} to reduce heap fragmentation.
        \item Optionally, integrate a counter in each node to measure recursion depth and further analyze performance.
    \end{itemize}
\end{itemize}

\end{document}
