\documentclass[12pt]{article}
\usepackage[utf8]{inputenc}
\usepackage{amsmath,amssymb}
\usepackage{enumitem}
\usepackage{hyperref}
\usepackage{geometry}
\geometry{margin=1in}

\title{Assignment: Abstract Math in C++ for AI \& Performance Systems}
\author{}
\date{}

\begin{document}

\maketitle

\section*{Overview}
This curriculum introduces math-rich ideas while progressively developing your C++ skills. Each stage reinforces learning through proofs, linear algebra, and practical modeling. Bonus “black diamond” challenges push the boundaries on performance and parallelization.

\section{Stage 1: Class and Function Synthesis --- Mathematical Objects in Code}
\subsection*{Theme: Modeling Abstract Math Concepts}

\subsubsection*{Project 1: Peano Arithmetic Synthesizer}
\begin{itemize}[leftmargin=*]
    \item \textbf{Goal:} Implement natural numbers using the Peano axioms (Zero, Successor).
    \item \textbf{Part 1:} Write a function-based model of addition and multiplication using recursion.
    \item \textbf{Part 2:} Develop a \texttt{PeanoNumber} class with operator overloads.
    \item \textbf{Bonus:} Use compile-time recursion with \texttt{constexpr} (template metaprogramming preview).
    \item \textbf{Rationale:} Reinforces recursion and inductive thinking (directly related to \emph{How to Prove It}).
\end{itemize}

\subsubsection*{Project 2: Vector Space Playground}
\begin{itemize}[leftmargin=*]
    \item \textbf{Goal:} Implement a minimal vector space class for 2D/3D.
    \item \textbf{Part 1:} Define a \texttt{Vector} class with dot product, scalar multiplication, and magnitude.
    \item \textbf{Part 2:} Enforce axioms via function contracts/assertions.
    \item \textbf{Part 3:} Implement transformations such as projection and extraction of orthogonal components.
    \item \textbf{Bonus:} Optimize the \texttt{dot()} function for float vectors using SIMD.
\end{itemize}

\section{Stage 2: Templates, Overloads, and Abstraction --- Building Generic Math Libraries}
\subsubsection*{Project 3: Template Matrix Library ($\mathbb{R}^n$)}
\begin{itemize}[leftmargin=*]
    \item \textbf{Goal:} Build a templated \texttt{Matrix<T, Rows, Cols>} class.
    \item \textbf{Part 1:} Implement basic operations: addition, scalar multiplication, and transpose.
    \item \textbf{Part 2:} Add support for operator overloading and slicing.
    \item \textbf{Part 3:} Develop a Gaussian elimination function.
    \item \textbf{Bonus:} Use \texttt{std::span} with alignment-aware storage and benchmark the performance.
\end{itemize}

\subsubsection*{Project 4: Symbolic Logic Expression Tree}
\begin{itemize}[leftmargin=*]
    \item \textbf{Goal:} Parse and evaluate propositional logic formulas.
    \item \textbf{Part 1:} Construct an \texttt{Expr} class hierarchy (including \texttt{And}, \texttt{Or}, \texttt{Not}, \texttt{Var}).
    \item \textbf{Part 2:} Recursively evaluate expressions based on a truth assignment.
    \item \textbf{Bonus:} Create a simplifier that applies De Morgan’s laws and eliminates double negations.
\end{itemize}

\section{Stage 3: Memory Management \& System Modeling --- Low-Level Behavior of Abstract Systems}
\subsubsection*{Project 5: Markov Chain Simulator}
\begin{itemize}[leftmargin=*]
    \item \textbf{Goal:} Model discrete Markov processes using dynamic memory (state machine).
    \item \textbf{Part 1:} Develop a \texttt{State} class with transitions and associated probabilities.
    \item \textbf{Part 2:} Implement Monte Carlo simulation of state transitions.
    \item \textbf{Bonus:} Compare performance between \texttt{std::vector} and a custom arena allocator.
\end{itemize}

\subsubsection*{Project 6: Multivariate Function Optimizer (Gradient Descent)}
\begin{itemize}[leftmargin=*]
    \item \textbf{Goal:} Implement a simple gradient descent system without autodiff.
    \item \textbf{Part 1:} Design a \texttt{Function<T>} class with evaluation and gradient estimation.
    \item \textbf{Part 2:} Optimize convex functions such as L2 and quadratic bowls.
    \item \textbf{Bonus:} Use OpenMP to parallelize the gradient estimation.
\end{itemize}

\section{Stage 4: Advanced Structures, Algebraic Systems --- Encoding Deep Mathematical Objects}
\subsubsection*{Project 7: Finite Field Arithmetic Library (GF(p))}
\begin{itemize}[leftmargin=*]
    \item \textbf{Goal:} Build modular arithmetic types to facilitate polynomial math.
    \item \textbf{Part 1:} Implement a \texttt{ModInt<p>} class with overloaded operators and the extended Euclidean algorithm.
    \item \textbf{Part 2:} Add polynomial division and compute the GCD over GF(p).
    \item \textbf{Bonus:} Integrate SIMD acceleration and lookup table optimizations.
\end{itemize}

\subsubsection*{Project 8: Hilbert Curve Encoder/Decoder}
\begin{itemize}[leftmargin=*]
    \item \textbf{Goal:} Develop a 2D and 3D spatial locality-preserving indexer.
    \item \textbf{Part 1:} Implement the recursive Hilbert curve algorithm from scratch.
    \item \textbf{Part 2:} Encapsulate the solution in a \texttt{HilbertMapper} class with unit tests.
    \item \textbf{Bonus:} Benchmark against a Z-order curve and explore cache-aware matrix block layouts.
\end{itemize}

\section{Stage 5: Probabilistic Modeling \& Inference --- Foundations of AI/ML in Systems Code}
\subsubsection*{Project 9: Bayesian Coin Inference Engine}
\begin{itemize}[leftmargin=*]
    \item \textbf{Goal:} Compute the posterior distribution of a coin's bias from observed flips.
    \item \textbf{Part 1:} Develop a \texttt{BetaDistribution} class with appropriate update rules.
    \item \textbf{Part 2:} Simulate updates and visualize convergence (export data as CSV).
    \item \textbf{Bonus:} Implement a Monte Carlo sampler using the Box-Muller transform for noise modeling.
\end{itemize}

\subsubsection*{Project 10: Kernels \& RKHS Prototyper}
\begin{itemize}[leftmargin=*]
    \item \textbf{Goal:} Implement common kernel functions and compute the Gram matrix.
    \item \textbf{Part 1:} Code Gaussian/RBF and polynomial kernels.
    \item \textbf{Part 2:} Classify points in toy datasets using the computed kernels.
    \item \textbf{Bonus:} SIMD-optimize the kernel matrix computation and measure its performance.
\end{itemize}

\section*{Optional Side Quests: Black Diamond Challenges}
\begin{itemize}[leftmargin=*]
    \item Write a unit test suite in pure C++.
    \item Use OpenMP or \texttt{std::thread} to parallelize workloads.
    \item Implement custom memory pools for allocation-heavy classes.
    \item Add CSV logging/profiling hooks to track runtime performance.
\end{itemize}

\end{document}
