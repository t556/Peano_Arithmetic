\documentclass{article}
\usepackage[utf8]{inputenc}
\usepackage{amsmath}
\usepackage{geometry}
\usepackage{xcolor}
\usepackage{listings}
\geometry{margin=1in}

\title{Big-O Complexity Cheat Sheet: AS/CR Primer}
\author{}
\date{}

\begin{document}

\maketitle

\section*{ Common Big-O Complexities (AS Table)}

\begin{tabular}{|c|l|l|}
\hline
\textbf{Big-O} & \textbf{Growth Rate}        & \textbf{Example Pattern} \\
\hline
\( O(1) \)     & Constant                    & Accessing \texttt{arr[i]}, hash map lookup \\
\( O(\log n) \) & Logarithmic                 & Binary search, tree traversal \\
\( O(n) \)     & Linear                      & Single loop over array \\
\( O(n \log n) \) & Linearithmic              & Merge sort, heap sort \\
\( O(n^2) \)   & Quadratic                   & Nested loops, naive matrix mult \\
\( O(2^n) \)   & Exponential                 & Recursive subset generation \\
\( O(n!) \)    & Factorial                   & Permutations, TSP brute-force \\
\hline
\end{tabular}

\vspace{1em}
\textbf{AS Tip:} Find loops and recursion — they’re your signal.  
\textbf{CR Tip:} Use scaling intuition: \textit{“what happens if I double n?”}

\section*{ How to Analyze Time Complexity (AS Steps)}

\begin{enumerate}
    \item Identify all loops and recursive calls.
    \item Estimate how many times each will run.
    \item Multiply nested operations.
    \item Keep the most dominant term (drop constants).
\end{enumerate}

\vspace{1em}
\textbf{Example:}
\begin{lstlisting}[language=C++]
for (int i = 0; i < n; i++)        // O(n)
  for (int j = 0; j < n; j++)      // O(n)
    doSomething();                 // O(1)
\end{lstlisting}

\textbf{Total: } \( O(n \times n) = O(n^2) \)

\section*{ CR Memory Hooks}

\begin{itemize}
    \item \( O(1) \): Pick a card from the top — instant.
    \item \( O(n) \): Flip through every page in a book.
    \item \( O(\log n) \): Keep splitting a phone book in half.
    \item \( O(n \log n) \): Sort a big deck by repeatedly merging halves.
    \item \( O(n^2) \): Everyone shakes hands with everyone else.
    \item \( O(2^n) \): Try every outfit combination from your wardrobe.
\end{itemize}

\section*{ Lower Bounds (Proven Limits)}

Some operations are provably impossible to improve beyond a certain point.

\begin{tabular}{|l|l|l|}
\hline
\textbf{Task}                  & \textbf{Best Possible} & \textbf{Why} \\
\hline
Searching in unsorted array   & \( O(n) \)             & Must check all elements \\
Searching in sorted array     & \( O(\log n) \)        & Binary search is optimal \\
Comparison sorting            & \( O(n \log n) \)      & Proven lower bound \\
Hash lookup (avg case)        & \( O(1) \)             & But worst-case \( O(n) \) \\
Matrix multiplication         & \( O(n^{2.3729}) \)    & No \( O(n \log n) \) known \\
Subset sum / TSP / SAT        & \( O(2^n) \)           & NP-complete; exponential only \\
\hline
\end{tabular}

\section*{ CR Visualization: Scaling Impact}

Assume an algorithm takes 1ms at \( n = 100 \).

\begin{tabular}{|c|c|}
\hline
\textbf{Complexity} & \textbf{Time at \( n = 10,000 \)} \\
\hline
\( O(1) \)           & 1ms \\
\( O(\log n) \)      & ~2ms \\
\( O(n) \)           & 100x → 100ms \\
\( O(n \log n) \)    & ~132ms \\
\( O(n^2) \)         & 10,000x → 10,000ms = 10s \\
\( O(2^n) \)         & \textit{unimaginable} \\
\hline
\end{tabular}

\section*{Tips for Choosing Algorithms}

\begin{itemize}
    \item Use \( O(1) \) or \( O(\log n) \) if data is indexed or sorted.
    \item Aim for \( O(n) \) if you must read all items.
    \item Accept \( O(n \log n) \) for sorting or divide-and-conquer.
    \item Avoid \( O(n^2) \)+ unless \( n \leq 1000 \).
    \item Avoid \( O(2^n) \)+ unless \( n \leq 20 \) or you're pruning search.
\end{itemize}

\section*{Final Rule of Thumb}

If your algorithm is:

\begin{itemize}
    \item \( O(1) \) or \( O(\log n) \): \textbf{scales beautifully}
    \item \( O(n) \) or \( O(n \log n) \): \textbf{usually safe}
    \item \( O(n^2) \) or worse: \textbf{danger zone beyond 10,000 items}
\end{itemize}

\newpage

\section*{Part 1 — Practice Problems}

Estimate the **worst-case time complexity** in Big-O for each function below. Assume input size is \( n \).

\vspace{1em}

\textbf{1. Single loop}
\begin{lstlisting}[language=C++]
void func1(int n) {
    for (int i = 0; i < n; ++i) {
        std::cout << i;
    }
}
\end{lstlisting}

\vspace{0.5em}
\textbf{2. Nested loop}
\begin{lstlisting}[language=C++]
void func2(int n) {
    for (int i = 0; i < n; ++i)
        for (int j = 0; j < n; ++j)
            std::cout << i << j;
}
\end{lstlisting}

\vspace{0.5em}
\textbf{3. Logarithmic growth}
\begin{lstlisting}[language=C++]
void func3(int n) {
    while (n > 1) {
        n = n / 2;
    }
}
\end{lstlisting}

\vspace{0.5em}
\textbf{4. Recursive Fibonacci (naive)}
\begin{lstlisting}[language=C++]
int fib(int n) {
    if (n <= 1) return n;
    return fib(n-1) + fib(n-2);
}
\end{lstlisting}

\vspace{0.5em}
\textbf{5. Merge sort-like pattern}
\begin{lstlisting}[language=C++]
void mergeSort(int arr[], int n) {
    if (n <= 1) return;
    int mid = n / 2;
    mergeSort(arr, mid);
    mergeSort(arr + mid, n - mid);
    // merge step (assume linear time)
}
\end{lstlisting}

\vspace{0.5em}
\textbf{6. Two separate loops}
\begin{lstlisting}[language=C++]
void func6(int n) {
    for (int i = 0; i < n; ++i)
        std::cout << i;
    for (int j = 0; j < n; ++j)
        std::cout << j;
}
\end{lstlisting}

\newpage

\section*{Part 2 — Answer Key and Explanations}

\textbf{1. Single loop} — \fbox{\( O(n) \)}

\textit{The loop runs exactly \( n \) times. Each step is \( O(1) \).}

---

\textbf{2. Nested loop} — \fbox{\( O(n^2) \)}

\textit{Inner loop runs \( n \) times per outer loop. Total steps: \( n \times n \).}

---

\textbf{3. Logarithmic growth} — \fbox{\( O(\log n) \)}

\textit{Each iteration halves \( n \). Number of steps is log base 2 of \( n \).}

---

\textbf{4. Recursive Fibonacci} — \fbox{\( O(2^n) \)}

\textit{Two recursive calls per level, forms an exponential tree.}

---

\textbf{5. Merge sort} — \fbox{\( O(n \log n) \)}

\textit{Divide: \( \log n \) levels; Merge: \( O(n) \) per level. Total: \( n \log n \).}

---

\textbf{6. Two separate loops} — \fbox{\( O(n) \)}

\textit{First loop is \( O(n) \), second is \( O(n) \). Add them → still \( O(n) \).}



\end{document}