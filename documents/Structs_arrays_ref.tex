\documentclass{article}
\usepackage[utf8]{inputenc}
\usepackage{amsmath,amsfonts}
\usepackage{listings}
\usepackage{xcolor}
\usepackage{geometry}
\geometry{margin=1in}

\definecolor{codebg}{rgb}{0.95,0.95,0.92}
\lstset{
  backgroundcolor=\color{codebg},
  basicstyle=\ttfamily\small,
  keywordstyle=\color{blue},
  commentstyle=\color{gray},
  stringstyle=\color{green!40!black},
  numbers=left,
  numberstyle=\tiny,
  breaklines=true,
  captionpos=b
}

\title{Quick Reference: Structs, Pointers, Arrays, and Memory in C/C++}
\author{}
\date{}

\begin{document}

\maketitle

\section*{Structs: Grouping Data}

A \texttt{struct} groups related variables under one name.

\begin{lstlisting}[language=C++]
struct Point {
    int x;
    int y;
};
\end{lstlisting}

\textbf{AS Syntax Pattern:}
\begin{itemize}
    \item Declare with \texttt{struct Name \{...\};}
    \item Use with \texttt{Point p; p.x = 10;}
\end{itemize}

\textbf{CR Visual:} Like a labeled box holding related slots: \texttt{x | y}

\vspace{0.5em}
\textbf{Bonus — C++ Style Struct with Constructor:}

\begin{lstlisting}[language=C++]
struct Point {
    int x, y;
    Point(int x_, int y_) : x(x_), y(y_) {}
};
\end{lstlisting}

\section*{Pointers: Memory Addresses}

A pointer \textit{stores the memory address} of a variable.

\begin{lstlisting}[language=C++]
int a = 42;
int* ptr = &a;     // ptr points to a
std::cout << *ptr; // dereference = 42
\end{lstlisting}

\textbf{AS Cheat Sheet:}
\begin{itemize}
    \item \texttt{*} declares a pointer: \texttt{int* p;}
    \item \texttt{\&} gets the address: \texttt{\&x}
    \item \texttt{*p} dereferences the pointer (gets the value)
\end{itemize}

\textbf{CR Visual:} \texttt{ptr → [42]} — like a cable connecting to a memory slot.

\section*{Arrays and Pointers}

\begin{lstlisting}[language=C++]
int arr[3] = {1, 2, 3};
int* p = arr;        // arr == &arr[0]
std::cout << p[1];   // prints 2
\end{lstlisting}

\textbf{Arrays decay to pointers.} You can treat the name of the array as a pointer.

\section*{Passing Pointers to Functions}

\textbf{Why:} Let a function access/modify the original variable or structure.

\begin{lstlisting}[language=C++]
void increment(int* x) {
    (*x)++;
}

int main() {
    int val = 10;
    increment(&val);  // pass pointer
    std::cout << val; // prints 11
}
\end{lstlisting}

\textbf{AS Tip:}
\begin{itemize}
    \item Passing by pointer lets you modify values in-place
\end{itemize}

\textbf{CR Analogy:} You gave the function your variable’s “address” — like handing over a key instead of a copy.

\section*{Pointers to Structs}

\begin{lstlisting}[language=C++]
Point pt(2, 3);
Point* ptr = &pt;
std::cout << ptr->x; // use '->' for member access
\end{lstlisting}

\textbf{Syntax Rule:} Use \texttt{->} to access members from a pointer to a struct.

\section*{Dynamic Memory: malloc/new}

Allocate memory manually during runtime.

\vspace{0.5em}
\textbf{C Style:}

\begin{lstlisting}[language=C]
int* data = (int*)malloc(sizeof(int) * 10);
data[0] = 42;
free(data);
\end{lstlisting}

\vspace{0.5em}
\textbf{C++ Style:}

\begin{lstlisting}[language=C++]
int* data = new int[10];
data[0] = 42;
delete[] data;
\end{lstlisting}

\textbf{For structs:}

\begin{lstlisting}[language=C++]
Point* p = new Point(5, 6); // constructor call
delete p;
\end{lstlisting}

\textbf{Important: Always free what you allocate!}

\section*{Memory Layout Insight (CR Mental Image)}

Imagine RAM as a giant warehouse with numbered lockers.

\begin{itemize}
    \item \texttt{int x = 5;} puts 5 in locker \#1000 (example).
    \item \texttt{int* p = \&x;} gives you the key to locker \#1000.
    \item \texttt{*p = 7;} writes 7 in locker \#1000.
\end{itemize}

\textbf{For arrays and structs:} memory is continuous and packed.

\section*{Putting It All Together: Node Chaining Example}

\begin{lstlisting}[language=C++]
struct Node {
    std::string key;
    int value;
    Node* next;
};

Node* head = new Node{"key1", 123, nullptr};
head->next = new Node{"key2", 456, nullptr};
\end{lstlisting}

\textbf{Memory layout:}
\[
\text{[head]} \rightarrow \text{[key1, 123]} \rightarrow \text{[key2, 456]} \rightarrow NULL
\]

This is the core of chained hash tables!

\section*{Summary Cheatsheet}

\begin{itemize}
    \item \texttt{struct S \{ ... \};} defines a custom type
    \item \texttt{S s;} or \texttt{S* sp = new S();}
    \item \texttt{*p} = value pointed to; \texttt{\&x} = address of x
    \item \texttt{malloc/free} = C style, \texttt{new/delete} = C++ style
    \item Pass pointers to functions for in-place modification
    \item Use \texttt{->} to access members via pointers
    \item Arrays and pointers are deeply connected
\end{itemize}

\end{document}
